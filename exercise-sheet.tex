\documentclass[a4paper]{article}

\usepackage{url}

\title{Using Git and GitHub effectively}
\begin{document}
\maketitle

The central repo will be:

\begin{quote}
\url{https://github.com/Research-Software-Kent/git-and-github-material}
\end{quote}

This contains a file \texttt{names-and-food.md} which we want to contain a list of everyone's names (you can use a fake name!) and a list of everyone's favourite foods (with duplicates removed), which are split into two sections.

The task is to update the file utilising forks, branches, and local changes propagated to the remote repository and then back again, propagating all the changes to our local repositories to reach \emph{eventual consistency}: everyone with the same local view of the files.

\paragraph{ \textsc{Part I}} You will need a GitHub account if you don't
already have one.

\begin{enumerate}
\item Make an issue on the central repo about your data being missing.
\item Fork the central repo and make a local copy (your choice of method).
\item Make a new branch for your fix.
\item Make your local updates to the file and split your change into two commits by using \texttt{git add -p}.

Use this as an opportunity to check you can also use:

\begin{enumerate}
\item \texttt{git stash} (i.e., make some local change then stash it
before it is staged, and pop it back).
\item \texttt{git reset} (i.e., stage a change then restore it to the index).
\end{enumerate}

Create two suitable commits referencing your original issue (with
the last \emph{closing it}).

If you do something wrong reset the commit and try again before pushing.

\item Push and make a PR to the central repo.
\end{enumerate}

\paragraph{\textsc{Part II} -- after the break}

After the break I will merge all the PRs and show conflict resolution
(there will probably be many). Then:

\begin{enumerate}
\item Update from the central repo back to your own (via GitHub interface).
\item Pull the updates back to your repo.
\item Merge locally into your \texttt{main} branch.
\end{enumerate}

\end{document}
